\documentclass[a4paper,oneside]{article}

\usepackage[frenchb]{babel}
\usepackage[utf8]{inputenc}  
\usepackage{graphicx}
\usepackage{url}
\usepackage{float}
\usepackage{color}
\usepackage{hyperref}
\usepackage{multicol}
\usepackage{latexsym}
\usepackage{amssymb}

\definecolor{dgreen}{RGB}{0,125,0}

\usepackage{geometry}
\geometry{left=3cm,right=3cm,top=3cm,bottom=2cm}

\title{\huge Memo CMake}
\author{L\'eo \textsc{Baudouin}}
\date{}

\begin{document}

{\Huge Memo CMake}
%----------------------------------------------------------------------
\newline

\noindent Pour plus d'informations, contactez moi à l'adresse : 
\href{mailto:baudouin.leo@gmail.com}{\nolinkurl{baudouin.leo@gmail.com}}.

\section{Aide en ligne}

\url{https://cmake.org/cmake/help/v3.14/}

%----------------------------------------------------------------------

\section{Fonctions de base}
\begin{itemize}
\item Mettre des commentaires :\\
  \textcolor{blue}{\verb?\#Commentaire ?}
\item Affiche un message lors de l'appel à CMake :\\
  \textcolor{blue}{\verb?MESSAGE([FATAL\_ERROR|WARNING|...] "message") ?}
\item Définir la version minimale de CMake :\\
  \textcolor{blue}{\verb?CMAKE\_MINIMUM\_REQUIRED(VERSION 2.6)?}
\end{itemize}

\subsection{Utiliser les variables}
\begin{itemize}
\item Définir une variable :\\
  \textcolor{blue}{\verb?SET(variable\_name "value")?}
\item Récupérer la valeur d'une variable :\\
  \textcolor{blue}{\verb?\$\{variable\_name\}?}
\item Créer des options :\\
  \textcolor{blue}{\verb?OPTION(variable\_name "Description" default\_value)?}
\item Lister les fichiers d'un dossier :\\
  \textcolor{blue}{\verb?FILE(GLOB variable pattern)?}\\
  Exemple : \verb?FILE(GLOB SRCS src/*.cpp)?
\end{itemize}

\subsection{Utiliser les cibles}
\begin{itemize}
\item Ajouter un exécutable :\\
  \textcolor{blue}{\verb?ADD\_EXECUTABLE( exec\_name sources )?}
\item Ajouter une bibliothèque :\\
  \textcolor{blue}{\verb?ADD\_LIBRARY( lib\_name [static|shared] sources )?}
\item Linker une cible avec une bibliothèque :\\
  \textcolor{blue}{\verb?TARGET\_LINK\_LIBRARIES( exec\_name libraries )?}%\linebreak
  %Exemple : \textcolor{blue}{\verb?TARGET\_LINK\_LIBRARIES( MyExecutable \$\{PROJECT\_NAME\})?}
\item Ajoute des dépendances entre les cibles :\\
  \textcolor{blue}{\verb?ADD\_DEPENDENCIES ?}
\item Ajouter une cible personnalisée :\\
  \textcolor{blue}{\verb?ADD\_CUSTOM\_TARGET ?}\\
  Exemple :
\begin{verbatim}
  ADD_CUSTOM_TARGET(Name [ALL] [command1 [args1...]]
  [COMMAND command2 [args2...] ...]
  [DEPENDS depend depend depend ... ]
  [WORKING_DIRECTORY dir]
  [COMMENT comment] [VERBATIM]
  [SOURCES src1 [src2...]]
  )
\end{verbatim}
\end{itemize}

\subsection{Utiliser des bibliothèques}
\begin{itemize}
\item Chercher un package :\\
  \textcolor{blue}{\verb?FIND\_PACKAGE( package\_name [version] [REQUIRED] ) ?}\\
  Initialise habituellement les variables :
  \begin{itemize}
  \item \textcolor{blue}{\verb?package\_name\_FOUND ?}
  \item \textcolor{blue}{\verb?package\_name\_INCLUDE\_DIRS ?}
  \item \textcolor{blue}{\verb?package\_name\_LIBRARY ?}
  \item \textcolor{blue}{\verb?package\_name\_LIBRARY\_DIRS ?}
  \end{itemize}
\item Utiliser un dossier contenant des bibliothèques :\\
  \textcolor{blue}{\verb?LINK\_DIRECTORIES( folder )?}
\item Ajouter un dossier à la liste des dossiers à inclure :\\
  \textcolor{blue}{\verb?INCLUDE\_DIRECTORIES( folder ) ?}
\end{itemize}

\subsection{Autre}
\begin{itemize}
\item Installer un fichier :\\
  \textcolor{blue}{\verb?INSTALL?}\\
  Exemple :
\begin{verbatim}
  INSTALL(FILES my_file
  DESTINATION output_folder
  PERMISSIONS OWNER_READ GROUP_READ WORLD_READ OWNER_WRITE
  )
\end{verbatim}
\item Remplacer les variables par leurs valeurs dans un fichier :\\
  \textcolor{blue}{\verb?CONFIGURE\_FILE(input\_file output\_file)?}\\
  On utilise généralement des fichiers ".in"
\item Ajouter une définition :\\
  \textcolor{blue}{\verb?ADD\_DEFINITION( "-DUSE\_MYLIB" )?}\\
  Vous pouvez utiliser dans votre code C/C++ :\\
  \textcolor{dgreen}{\verb?\#ifdef USE\_MYLIB ?}
\item Exécuter une commande personnalisée :\\
  \textcolor{blue}{\verb?ADD\_CUSTOM\_COMMAND ?}\\
  Exemple :
\begin{verbatim}
  ADD_CUSTOM_COMMAND(OUTPUT output1 [output2 ...]
  COMMAND command1 [ARGS] [args1...]
  [COMMAND command2 [ARGS] [args2...] ...]
  [MAIN_DEPENDENCY depend]
  [DEPENDS [depends...]]
  [IMPLICIT_DEPENDS <lang1> depend1 ...]
  [WORKING_DIRECTORY dir]
  [COMMENT comment] [VERBATIM] [APPEND]
  )
\end{verbatim}
\item Exécuter un programme externe :\\
  \textcolor{blue}{\verb?EXEC\_PROGRAM ?}\\
  Exemple :
\begin{verbatim}
  EXEC_PROGRAM(Executable [directory in which to run]
  [ARGS <arguments to executable>]
  [OUTPUT_VARIABLE <var>]
  [RETURN_VALUE <var>]
  )
\end{verbatim}
\end{itemize}

\subsection{Test unitaires}
\begin{itemize}
\item Active les tests unitaires :\\
  \textcolor{blue}{\verb?ENABLE\_TESTING()?}
\item Ajouter un test unitaire :\\
  \textcolor{blue}{\verb?ADD\_TEST( test\_name exec\_name [arguments])?}\\
  Nécessite d'appeler une fois la fonction \textcolor{blue}{\verb?ENABLE\_TESTING()?}
\end{itemize}

\section{Variables de base}
\begin{itemize}
\item Définir le projet :\\
  \textcolor{blue}{\verb?SET(PROJECT\_NAME name)?}\\
  \textcolor{blue}{\verb?SET(PROJECT\_DESCRIPTION "")?}\\
  \textcolor{blue}{\verb?SET(PROJECT\_URL "")?}\\
  \textcolor{blue}{\verb?SET(PROJECT\_VERSION "0.0.1")?}\\
  \textcolor{blue}{\verb?SET(PROJECT\_REQUIREMENTS "")?}
\item Dossier contenant le premier CMakeLists.txt :\\
  \textcolor{blue}{\verb?CMAKE\_SOURCE\_DIR ?}
\item Dossier contenant le CMakeLists.txt courant :\\
  \textcolor{blue}{\verb?CMAKE\_CURRENT\_SOURCE\_DIR ?}
\item Dossier contenant le projet courant :\\
  \textcolor{blue}{\verb?PROJECT\_SOURCE\_DIR  ?}
\item Dossier de sortie du premier CMakeLists.txt :\\
  \textcolor{blue}{\verb?CMAKE\_BINARY\_DIR ?}
\item Dossier de sortie du CMakeLists.txt courant :\\
  \textcolor{blue}{\verb?CMAKE\_CURRENT\_BINARY\_DIR ?}
\item Dossier de sortie du projet courant :\\
  \textcolor{blue}{\verb?PROJECT\_BINARY\_DIR ?}
\item Dossier d'installation :\\
  \textcolor{blue}{\verb?CMAKE\_INSTALL\_PREFIX ?}
\item Liste des dossiers pour FIND\_PATH() :\\
  \textcolor{blue}{\verb?CMAKE\_INCLUDE\_PATH ?}
\item Liste des dossiers pour FIND\_LIBRARY() :\\
  \textcolor{blue}{\verb?CMAKE\_LIBRARY\_PATH ?}
\item Liste des dossiers pour FIND\_PACKAGE(), FIND\_PATH(), FIND\_PROGRAM() et FIND\_LIBRARY() :\\
  \textcolor{blue}{\verb?CMAKE\_PREFIX\_PATH ?}
\item Options de compilation :\\
  \textcolor{blue}{\verb?CMAKE\_CXX\_FLAGS ?},
  \textcolor{blue}{\verb?CMAKE\_CXX\_FLAGS\_DEBUG ?},
  \textcolor{blue}{\verb?CMAKE\_CXX\_FLAGS\_RELEASE ?}
\item Choisir le compilateur :\\
  \textcolor{blue}{\verb?CMAKE\_CXX\_COMPILER?}
\item Compiler les bibliothèques en mode dynamique :\\
  \textcolor{blue}{\verb?SET(BUILD\_SHARED\_LIBS ON)?}
\item Définir le dossier de sortie des exécutables :\\
  \textcolor{blue}{\verb?RUNTIME\_OUTPUT\_DIRECTORY?}
\item Définir le dossier de sortie des bibliothèques :\\
  \textcolor{blue}{\verb?LIBRARY\_OUTPUT\_DIRECTORY?}
\end{itemize}

\section{Programmation}
\subsection{Fonction IF}
\begin{verbatim}

  SET( number 4 )
  IF( number GREATER 10 )
    MESSAGE( "The number ${number} is too large." )
  ELSE()
    MESSAGE( "The number ${number} is ok." )
  ENDIF()

  SET( x )
  IF(DEFINED x)
    MESSAGE("This will never be printed")
  ENDIF()
\end{verbatim}

\subsection{Fonction FOREACH}
\begin{verbatim}
  foreach(func ${funcs})
    set(func_name test_${func})
    eval("${func_name}(\"Test\")")
  endforeach(func)
\end{verbatim}

\end{document}
