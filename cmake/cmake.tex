\documentclass{beamer}
%\usetheme{Madrid}
%\usetheme{Boadilla}
%\usetheme{default}
%\usetheme{Warsaw}
%\usetheme{Bergen}
%\usetheme{Frankfurt}
\usetheme{Darmstadt}

%\usecolortheme{seahorse}
%\usecolortheme{beaver}
\setbeamercolor{normal text}{fg=white}
\setbeamertemplate{background canvas}[vertical shading] [top=black!95,bottom=black!65]

\definecolor{mypurple}{RGB}{207,78,64}
\usecolortheme[named=mypurple]{structure}

\definecolor{myorange}{RGB}{255,235,190}
\beamerboxesdeclarecolorscheme{orange}{orange}{myorange}


%\definecolor{commandcolor}{RGB}{11,95,65}
\definecolor{commandcolor}{RGB}{111,195,165}

\setbeamertemplate{footline}[page number]
%\setbeamercovered{transparent}
\setbeamercovered{invisible}
\setbeamertemplate{navigation symbols}{}

%\usepackage{musixtex}
\usepackage{multimedia}
\usepackage{graphicx}
\usepackage[utf8x]{inputenc}
%\usepackage[T1]{fontenc}
\usepackage[frenchb]{babel} 
%\usepackage[all]{xy}
%\usepackage{multirow}
%\usepackage{lmodern}
\usepackage{subfigure}
%\usepackage{ulem}
\usepackage{url}
\usepackage{hyperref}
\usepackage{verbatim}
\usepackage{xspace}
\usepackage{color}
%\usepackage{xcolor}
\usepackage{rotating}
\usepackage{multicol}
\usepackage{fancyvrb}
\usepackage{listings}

\definecolor{mypurple}{RGB}{207,78,64}
\usecolortheme[named=mypurple]{structure}

\definecolor{myorange}{RGB}{255,235,190}
\beamerboxesdeclarecolorscheme{orange}{orange}{myorange}

\definecolor{myorange}{RGB}{255,235,190}


\lstset{
    %frame=tb, % draw a frame at the top and bottom of the code block
    tabsize=4, % tab space width
    showstringspaces=false, % don't mark spaces in strings
    numbers=left, % display line numbers on the left
    commentstyle=\color{green}, % comment color
    keywordstyle=\color{myorange}, % keyword color
    stringstyle=\color{mypurple} % string color
}


\definecolor{dgreen}{RGB}{0,125,0}


\usepackage{tikz}
\usetikzlibrary{trees}

\setbeamertemplate{caption}[numbered] 

\newcommand{\setframetitle}[1]{\begin{center}
    \huge \textbf{#1}
\end{center}}

%% --------------

\title{CMake}
\subtitle{Atelier d'aide à la programmation}
\author{L\'eo \textsc{Baudouin}}
\institute{
  {\url{baudouin.leo @ gmail.com}}
}
\date{03-04 juin 2019}

%% --------------

\begin{document}

\begin{frame}
  \titlepage
\end{frame}

\section{Makefile}
\subsection{}

\begin{frame}[fragile]{Télécharger les exemples}
  \begin{block}{Cloner le dép\^ot suivant}
    \url{https://github.com/lbaudouin/module-cmake.git}
  \end{block}
\end{frame}

\begin{frame}[fragile]{Exemple 1 : Création d'un \textit{Hello World} en C++}

  \begin{block}{Fichier : \textbf{hello.cpp}}
  \begin{lstlisting}[language=C++]
#include <iostream>
int main()
{
  std::cout << "Hello world !" << std::endl;
  return 0;
}
\end{lstlisting}
  \end{block}
  \pause

  \begin{block}{Compiler et lancer le programme}
    \textcolor{commandcolor}{\verb?g++ hello.cpp -o hello?}\linebreak
    \textcolor{commandcolor}{\verb?./hello?}
  \end{block}

\end{frame}

\begin{frame}[fragile]{Exemple 2 : Le Makefile}
  Permet de compiler automatiquement en faisant \textbf{make}

  \begin{exampleblock}{Template d'un Makefile}
  \begin{lstlisting}[language=make]
cible: dependances
  commandes
\end{lstlisting}
  \end{exampleblock}
  \pause

  \begin{block}{Fichier : \textbf{Makefile}}
  \begin{lstlisting}[language=make]
hello: main.c hello.c
  gcc -o hello hello.c main.c
\end{lstlisting}
  \end{block}
  \pause

  \begin{block}{Compiler et lancer le programme}
    \textcolor{commandcolor}{\verb?make?}\linebreak
    \textcolor{commandcolor}{\verb?./hello?}
  \end{block}
  
\end{frame}

\begin{frame}[fragile]{Exemple 3 : Le Makefile}

  \begin{block}{Fichier : \textbf{Makefile}}
  \begin{lstlisting}[language=make]
hello: hello.o main.o
  gcc -o hello hello.o main.o

hello.o: hello.c
  gcc -o hello.o -c hello.c -Wall

main.o: main.c hello.h
  gcc -o main.o -c main.c -Wall

clean:
  rm -rf *.o
\end{lstlisting}
  \end{block}
  
\end{frame}

\begin{frame}[fragile]{Exemple 4 : Le Makefile avancé}

  \begin{block}{Fichier : \textbf{Makefile}}
  \begin{lstlisting}[language=make]
CC = gcc
CFLAGS = -W -Wall

all: hello.o main.o
  $(CC) -o hello $^

%.o: %.c
  $(CC) -c $< -o $@ $(CFLAGS)

clean:
  rm -rf *.o
\end{lstlisting}
  \end{block}
  
\end{frame}


\begin{frame}[fragile]{Exemple 5 : CMake}

  \begin{block}{Fichier : \textbf{CMakeLists.txt}}
  \begin{lstlisting}[language=make]
ADD_EXECUTABLE(hello main.c hello.c)
\end{lstlisting}
  \end{block}
  
  \begin{block}{Compiler et lancer le programme}
    \textcolor{commandcolor}{\verb?mkdir build; cd build?}\linebreak
    \textcolor{commandcolor}{\verb?cmake ..?}\linebreak
    \textcolor{commandcolor}{\verb?make?}\linebreak
    \textcolor{commandcolor}{\verb?./hello?}
  \end{block}

  \begin{exampleblock}{A voir}
    Ouvrir le fichier Makefile généré
  \end{exampleblock}

\end{frame}


\begin{frame}[fragile]{Astuce}
  \begin{block}{CMake}
    \begin{itemize}
    \item Utilisation d'un dossier de \textbf{build}
    \item Utilisation des variables : \verb?cmake -DCMAKE_BUILD_TYPE=Release ..?

    \item Gestion des variables/options avec \textbf{ccmake} ou \textbf{cmake-gui}
    \end{itemize}
  \end{block}

  \begin{block}{Make}
    \begin{itemize}
    \item Utiliser plusieurs threads : \verb?make -j 5?
    \item Forcer la recompilation : \verb?make -B?
    \item Passer les erreurs : \verb? make -k?
    \end{itemize}
  \end{block}
\end{frame}

\section{CMake pour un projet complet}
\subsection{}

\begin{frame}[fragile]{Utilisation de CMake}

  \begin{block}{Cloner le dép\^ot suivant}
    \url{https://gitlab.com/lbaudouin/libempty.git}
  \end{block}

\end{frame}

\begin{frame}[fragile]{Arborescence}
  \begin{scriptsize}
\begin{verbatim}
|-- bin                        
|   |-- CMakeLists.txt       < Règles de compilations des binaires
|   |-- main.cpp             < Source d'un binaire
|-- CMakeLists.txt           < Règles générales
|-- doc                      
|   |-- doc.tag              < Fichier de documentation
|   |-- Doxyfile.in          < Règle de la documentation
|   |-- style.css            < Style pour la documentation
|-- include                  
|   |-- my_class.h           < Fichiers d'en-tête
|-- LibraryConfig.cmake.in   < Fichier de configuration pour cmake
|-- Library.pc.in            < Fichier de configuration pour pkg-config
|-- LICENSE                  < Licence du projet
|-- README                   < Aide pour l'utilisateur
|-- src
|   |-- my_class.cpp         < Sources du projet        
|-- unitTesting              < Tests unitaires
    |-- CMakeLists.txt       < Règles de compilations des tests unitaires
    |-- TestPass.cpp         < Test unitaire pour une seule fonctionnalité
\end{verbatim}
  \end{scriptsize}
\end{frame}

\begin{frame}{Compilation}
  \begin{exampleblock}{Réaliser les étapes suivantes :}
    \begin{enumerate}
    \item \verb?mkdir build; cd build?
    \item \verb?cmake ..?
    \item \verb?make \&\& make install? 
    \end{enumerate}
  \end{exampleblock}
  \pause
  \begin{exampleblock}{}
    \begin{enumerate}
      \setcounter{enumi}{4}
    \item \verb?ccmake ..?
    \item Modifier les options suivantes :\\ \begin{scriptsize}(BUILD\_TEST, CMAKE\_BUILD\_TYPE,  CMAKE\_INSTALL\_PREFIX)\end{scriptsize}
    \item \verb?make install?
    \item \verb?make test?
    \end{enumerate}
  \end{exampleblock}
  \pause
  \begin{exampleblock}{}
    \begin{enumerate}
      \setcounter{enumi}{8}
    \item \verb?make doc?
    \end{enumerate}
  \end{exampleblock}
\end{frame}

\section{Création du CMakeLists.txt}
\subsection{}
\begin{frame}{Création du CMakeLists.txt}
  \begin{exampleblock}{Objectifs}
    \begin{itemize}
    \item Compiler une bibliothèque (library)
    \item Compiler des binaires
    \item Compiler des tests unitaires
    \end{itemize}
  \end{exampleblock}
  \pause
  \begin{exampleblock}{}
    \begin{itemize}
    \item Installer la bibliothèque et les binaires
    \item Installer les fichiers de configuration
    \item Générer la documentation
    \item Ajouter des options de compilation
    \end{itemize}
  \end{exampleblock}
\end{frame}

\begin{frame}[fragile]{Fonctions}
  \begin{block}{Généralités}
    \begin{itemize}
    \item Mettre des commentaires :\linebreak
      \textcolor{cyan}{\verb?\#Commentaire ?}
    \item Version minimal de CMake :\linebreak
      \textcolor{cyan}{\verb?CMAKE\_MINIMUM\_REQUIRED(VERSION 2.6)?}
    \item Définir un projet :\linebreak
      \textcolor{cyan}{\verb?PROJECT(project\_name)?}
    \item Définir une variable :\linebreak
      \textcolor{cyan}{\verb?SET(variable\_name "value")?}
    \item Récupérer la valeur d'une variable :\linebreak
      \textcolor{cyan}{\verb?\$\{variable\_name\}?}
    \item Créer des options :\linebreak
      \textcolor{cyan}{\verb?OPTION(variable\_name "Description" default\_value)?}
    \end{itemize}
  \end{block}
\end{frame}

\begin{frame}{Consignes}

  \begin{exampleblock}{Créer un projet}
    \begin{itemize}
    \item Cloner le dépôt 
    \url{https://github.com/lbaudouin/cmake-demo.git}
    \item Définir la version minimal de cmake (ici 2.6)
	\item Définir un nouveau projet \og module \fg
    \end{itemize}
  \end{exampleblock}

\end{frame}

%OPTION(BUILD_SHARED_LIBRARY "Build the shared library." TRUE)
%OPTION(INSTALL_BINARIES "Install binaries." FALSE)
%OPTION(ENABLE_TEST "Enable test" FALSE)

\begin{frame}[fragile]{Fonctions}
  \begin{block}{Créer des cibles}
    \begin{itemize}
    \item Ajouter un exécutable :\linebreak
      \textcolor{cyan}{\verb?ADD\_EXECUTABLE( exec\_name sources )?}
    \item Ajouter une bibliothèque :\linebreak
      \textcolor{cyan}{\verb?ADD\_LIBRARY( lib\_name [static|shared] sources )?}
    \item Linker une cible avec une bibliothèque :\linebreak
      \textcolor{cyan}{\verb?TARGET\_LINK\_LIBRARIES( exec\_name libraries )?}%\linebreak
      %Exemple : \textcolor{cyan}{\verb?TARGET\_LINK\_LIBRARIES( MyExecutable \$\{PROJECT\_NAME\})?}
    \item Utiliser un dossier contenant des bibliothèques :\linebreak
      \textcolor{cyan}{\verb?LINK\_DIRECTORIES( folder )?}
    \end{itemize}
  \end{block}

  \begin{block}{Sous-dossier}
    \begin{itemize}
    \item Inclure un sous-dossier :\linebreak
      \textcolor{cyan}{\verb?ADD\_SUBDIRECTORY( folder )?}
    \end{itemize}
  \end{block}
\end{frame}

\begin{frame}{Consignes}
  \begin{exampleblock}{Créer les cibles}
    \begin{itemize}
    \item Ajouter les sources à une variable en utilisant la fonction FILE
    \item Créer la bibliothèque
    \item Créer le binaire \og demo \fg
    \item Créer les tests unitaires
    \end{itemize}
  \end{exampleblock}
\end{frame}

\subsection{Chercher une bibliothèque}
\begin{frame}[fragile]{Fonctions}
  \begin{block}{Utiliser une bibliothèque}
    \begin{itemize}
    \item Chercher une bibliothèque :\linebreak
      \textcolor{cyan}{\verb?FIND\_PACKAGE(package\_name [version] [EXACT] [REQUIRED])?}\\
      Cherche les fichiers :\\
      \begin{itemize}
      \item \verb?package_nameConfig.cmake ?
      \item \verb?package_name-config.cmake ?
      \end{itemize}
    \item Savoir si la bibliothèque a été trouvée :\linebreak
      \textcolor{cyan}{\verb?package\_name\_FOUND?}
    \item Ajouter des dossier d'includes :\linebreak
      \textcolor{cyan}{\verb?include\_directories( <directory> )?}
    \end{itemize}
  \end{block}
\end{frame}

\begin{frame}[fragile]{Fonctions}
  \begin{block}{Utiliser une bibliothèque}
    \begin{itemize}
    \item \verb?XXX_FOUND? :\\
      Passe à TRUE si la bibliothèque XXX a été trouvée
    \item \verb?XXX_YYY_FOUND? :\\
      Passe à TRUE si le composant YYY de XXX a été trouvé
    \item \verb?XXX_INCLUDE_DIRS? :\\
      Liste les dossiers d'include
    \item \verb?XXX_LIBRARIES? :\\
      Liste les fichiers contenant les bibliothèques (chemin complet)
    \item \verb?XXX_LIBRARY_DIRS? : (Optionnel)\\
      Liste les dossiers contenant les bibliothèques
    \item \verb?XXX_DEFINITIONS? :\\
      Liste les définitions pour la bibliothèque
    \end{itemize}
  \end{block}
\end{frame}

%XXX_FIND_VERSION       = full requested version string
%XXX_FIND_VERSION_MAJOR = major version if requested, else 0
%XXX_FIND_VERSION_MINOR = minor version if requested, else 0
%XXX_FIND_VERSION_PATCH = patch version if requested, else 0
%XXX_FIND_VERSION_TWEAK = tweak version if requested, else 0
%XXX_FIND_VERSION_COUNT = number of version components, 0 to 4
%XXX_FIND_VERSION_EXACT = true if EXACT option was given

\begin{frame}[fragile]{Fonctions}
  \begin{block}{Utiliser une bibliothèque avec un fichier .pc}
\begin{verbatim}
  SET(XXX_FOUND FALSE)
  FIND_PACKAGE(PkgConfig REQUIRED)
  IF(PKG_CONFIG_FOUND)
    PKG_CHECK_MODULES(XXX REQUIRED xxx>=1.8.0)
  ENDIF()

  IF(XXX_FOUND)
    MESSAGE("XXX found")
  ELSE()
    MESSAGE("XXX>=1.8.0 not found")
  ENDIF()
\end{verbatim}
  \end{block}
\end{frame}


\begin{frame}{Consignes}
  \begin{exampleblock}{Générer la doc}
    \begin{itemize}
    \item Chercher le package \textbf{Doxygen} avec \verb?FIND\_PACKAGE ?
    \item Tester si Doxygen a été trouvé
    \item Configurer le fichier \verb?\$\{CMAKE\_SOURCE\_DIR\}/doc/Doxyfile.in? et le sortir dans \verb?\$\{PROJECT\_BINARY\_DIR\}/doc/Doxyfile?
    \item Créer une cible personnalisée "doc" exécutant \verb?\$\{DOXYGEN\_EXECUTABLE\}?
    \end{itemize}
  \end{exampleblock}
\end{frame}

%-------------------------------------------------------------------

\end{document} 

%-------------------------------------------------------------------

%\transdissolve[duration=0.25]
%
%\begin{exampleblock}{Avantages}
%\end{exampleblock}
%
%\begin{alertblock}{Inconvénients}
%\end{alertblock}
