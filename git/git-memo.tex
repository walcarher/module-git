\documentclass[a4paper,oneside,twocolumn]{article}

\usepackage[frenchb]{babel}
\usepackage[utf8]{inputenc}  
\usepackage{graphicx}
\usepackage{url}
\usepackage{float}
\usepackage{color}
\usepackage{hyperref}
\usepackage{multicol}
\usepackage{latexsym}
\usepackage{amssymb}
%\usepackage{picins}

\usepackage{geometry}
\geometry{left=3cm,right=3cm,top=3cm,bottom=2cm}

\newcommand{\gitcommand}[2]{
\noindent\verb?#1?\linebreak
#2}


\begin{document}
\begin{center}
	\huge Memo - Git
\end{center}

%----------------------------------------------------------------------

\vspace{-3mm}
\section{Création}
\vspace{-2mm}

\noindent\verb?git init?\linebreak
$\triangleright$ initialise un dépôt local

\noindent\verb?git clone <url>?\linebreak
$\triangleright$ clone un dépôt existant

\vspace{-3mm}
\section{Information}
\vspace{-2mm}

\noindent\verb?git status?\linebreak
$\triangleright$ initialise un dépôt local

\noindent\verb?git diff [<file>]?\linebreak
$\triangleright$ affiche les différence entre l'espace de travail et la zone indexée

\noindent\verb?git diff --cached [<file>]?\linebreak
$\triangleright$ affiche les différence entre la zone indexée et le dépôt local

\noindent\verb?git diff <commit> [<file>]?\linebreak
$\triangleright$ affiche les différence entre l'espace de travail et le dernier commit

\noindent\verb?git log -p [<file>]?\linebreak
$\triangleright$ affiche l'historique avec les diff

\noindent\verb?git blame <file>?\linebreak
$\triangleright$ affiche les dernière modifications et auteurs ligne par ligne

\vspace{-3mm}
\section{Branche}
\vspace{-2mm}

\noindent\verb?git checkout <branch>?\linebreak
$\triangleright$ bascule l'espace de travail sur une branche

\noindent\verb?git checkout -b <new branch>?\linebreak
$\triangleright$ crée et bascule sur une nouvelle branche

\noindent\verb?git branch -d <old branch>?\linebreak
$\triangleright$ supprime l'ancienne branche, -r pour supprimer sur le serveur en même temps

\noindent\verb?git merge <branch>?\linebreak
$\triangleright$ fusionne la branche spécifiée dans la branche courante

\vspace{-3mm}
\section{Remisage}
\vspace{-2mm}

\noindent\verb?git stash save <mesage>?\linebreak
$\triangleright$ remise les modifications de l'espace de travail et fait \emph{reset}

\noindent\verb?git stash pop?\linebreak
$\triangleright$ dépile les modifications de la remise sans l'espace de travail

\noindent\verb?git stash list?\linebreak
$\triangleright$ list de contenu de la remise

\vspace{-3mm}
\section{Zone indexée}
\vspace{-2mm}

\noindent\verb?git add <file>?\linebreak
$\triangleright$ ajoute un fichier à la zone indexée

%\noindent\verb?git add -p?\linebreak
%$\triangleright$ ajoute de manière interactive  les modifications de l'espace de travail

\noindent\verb?git reset <file>?\linebreak
$\triangleright$ enlève le fichier de la zone indexée


\vspace{-3mm}
\section{Commit}
\vspace{-2mm}

\noindent\verb?git commit?\linebreak
$\triangleright$ valide la zone indexée dans le dépôt local

%\noindent\verb?git commit -p?\linebreak
%$\triangleright$ valide de manière interactive les modifications de l'espace de travail

\noindent\verb?git reset --hard?\linebreak
$\triangleright$ remet l'espace de travail ainsi que la zone indexée das l'état du dépôt et donc efface les modifications non commitées

\noindent\verb?git revert <commit>?\linebreak
$\triangleright$ applique les modification inverse au commit sélectionné

\noindent\verb?git cherry-pick <commit> ?\linebreak
$\triangleright$ applique un commit sur une autre branche

\vspace{-3mm}
\section{Dépôt distant}
\vspace{-2mm}

\noindent\verb?git remote add <remote> <url>?\linebreak
$\triangleright$ ajoute un dépôt distant nommé <remote>

\noindent\verb?git fetch <remote>?\linebreak
$\triangleright$ synchronise avec un dépôt distant% et ses références locales

\noindent\verb?git pull <remote> <branch>?\linebreak
$\triangleright$ récupère une branche du dépôt distant et la fusionne dans la branche courante

\noindent\verb?git pull --rebase <remote> <branch>?\linebreak
$\triangleright$ idem mais en rebasant la branche courante au lieu de fusionner

\noindent\verb?git push <remote> <branch>?\linebreak
$\triangleright$ envoie les commits de la branche sur le dépôt distant

\noindent\verb?git push <remote> <local_branch>:<remote_branch>?\linebreak
$\triangleright$ envoie la branche locale vers la branche distante

\noindent\verb?git push <remote>:<old_branch>?\linebreak
$\triangleright$ supprime la branche du dépôt distant

\noindent\verb?git push --tags?\linebreak
$\triangleright$ envoie les tags vers le dépôt distant

\vspace{-3mm}
\section{Rebase}
\vspace{-2mm}

\noindent\verb?git rebase master?\linebreak
$\triangleright$ se place au dernier commit de la branche master et ré-applique les commits de la branche courante

\noindent\verb?git rebase -i HEAD~3?\linebreak
$\triangleright$ liste les 3 derniers commits avant de les ré-appliquer, permettant de les modifier

\noindent\verb?git rebase --abort?\linebreak
$\triangleright$ annule le rebase en cours


\vspace{-3mm}
\section{Tag}
\vspace{-2mm}

\noindent\verb?git tag?\linebreak
$\triangleright$ liste les tags du dépôt local

\noindent\verb?git tag <new_tag>?\linebreak
$\triangleright$ crée un tag sur le dernier commit

\vspace{-3mm}
\section{Aide}
\vspace{-2mm}

\noindent\verb?man git-<command>?\linebreak
$\triangleright$ Aide sur la commande \emph{command}

\end{document}
